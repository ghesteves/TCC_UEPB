%%-----------------------------------------------------------------------
% O argumento chapter=TITLE define que todos os títulos de capítulos
% saiam em caixa alta (maiúsculas). Se não quiser caixa alta, basta
% remover este argumento.
\documentclass[article,12pt,oneside,a4paper,chapter=TITLE,
			   english,brazil]{abntex2}


%%-----------------------------------------------------------------------
% Carregando a padronização para a capa, contra-capa e folha de
% assinaturas adotada pela UEPB
\usepackage{TCC_UEPB}


%%-----------------------------------------------------------------------
% Pacotes básicos 
\usepackage{mathptmx}		% Usa a fonte Times New Roman
\usepackage[T1]{fontenc}	% Selecao de codigos de fonte.
\usepackage[utf8]{inputenc}	% Cod. do doc. (conversão autom. dos acentos)
\usepackage{lastpage}		% Usado pela Ficha catalográfica
\usepackage{indentfirst}	% Indenta o primeiro parágrafo de cada seção.
\usepackage{color}			% Controle das cores
\usepackage{graphicx}		% Inclusão de gráficos
\usepackage{microtype} 		% para melhorias de justificação
\usepackage{pdfpages}		% para a inclusão de documentos em PDF.


%%-----------------------------------------------------------------------
% Pacotes adicionais, para definir ambientes para definição, teorema 
% e axioma. Outros ambientes podem ser definidos da mesma forma...
\usepackage{amssymb}     % qed
\usepackage{amsthm}      % Teoremas
\usepackage{amsmath}     % Para o ambiente align (alinhar equações)
\usepackage{thmtools}    % Front end para amsthm (\declaretheorem)


%%-----------------------------------------------------------------------
% Definição de ambientes para definição, teorema, etc...
\declaretheorem[style=definition,name=Definição,qed=\textemdash]{definicao}
\declaretheorem[style=plain,name=Teorema,qed=\textnormal{\textemdash}]{teorema}
\declaretheorem[style=plain,name=Axioma,qed=\textnormal{\textemdash}]{axioma}
\declaretheorem[style=plain,name=Lema,qed=\textnormal{\textemdash}]{lema}


%%-----------------------------------------------------------------------
%% Pacotes de citações
\usepackage[brazilian,hyperpageref]{backref} % Página citada na bibliog.
\usepackage[alf]{abntex2cite}				 % Citações padrão ABNT



%%-----------------------------------------------------------------------
%% CONFIGURAÇÕES DE PACOTES
%%-----------------------------------------------------------------------


%%-----------------------------------------------------------------------
% Configurações do pacote backref Usado sem a opção hyperpageref 
% de backref
\renewcommand{\backrefpagesname}{Citado na(s) página(s):~}
% Texto padrão antes do número das páginas
\renewcommand{\backref}{}   % Define os textos da citação
\renewcommand*{\backrefalt}[4]{
	\ifcase #1 %
	Nenhuma citação no texto.%
	\or
	Citado na página #2.%
	\else
	Citado #1 vezes nas páginas #2.%
	\fi}%


%%------------------------------------------------------------------
% Configurações de aparência do PDF final
\makeatletter
\hypersetup{       % informações do PDF
	pdftitle={\@title}, 
	pdfauthor={\@author},
	pdfsubject={\imprimirpreambulo},
	colorlinks=true,  % false: box links; true: color links
	linkcolor=black,  % color of internal links
	citecolor=black,  % color of links to bibliography
	filecolor=black,  % color of file links
	urlcolor=black,   % color of url links
	bookmarksdepth=4
}
\makeatother


%%------------------------------------------------------------------
% Cria uma nova série (de símbolos) para footnotes
\newcounter{savefootnote}
\newcounter{symfootnote}
\newcommand{\symfootnote}[1]{%
	\setcounter{savefootnote}{\value{footnote}}%
	\setcounter{footnote}{\value{symfootnote}}%
	\ifnum\value{footnote}>8\setcounter{footnote}{0}\fi%
	\let\oldthefootnote=\thefootnote%
	\renewcommand{\thefootnote}{\fnsymbol{footnote}}%
	\footnote{#1}%
	\let\thefootnote=\oldthefootnote%
	\setcounter{symfootnote}{\value{footnote}}%
	\setcounter{footnote}{\value{savefootnote}}%
}


%%------------------------------------------------------------------
% Numeração das páginas no topo à direita
\usepackage{fancyhdr}
\pagestyle{fancy}
\pagestyle{fancyplain}
\fancyhf{}
\lhead{}
\rhead{\thepage}
\rfoot{}
\renewcommand{\headrulewidth}{0pt}


%%-----------------------------------------------------------------------
% Informações de dados para CAPA, FOLHA DE ROSTO e FOLHA DE APROVAÇÃO
\titulo{Título do trabalho: subtítulo}   % Subtítulo apenas se houver
\tituloestrangeiro{Work title: subtitle} % Title (English - opcional)
\autor{Nome do Aluno}                    % Digite seu nome aqui
\local{Campina Grande - PB}
\data{Ano}                               % Na data coloque apenas o ano
\orientador[Prof.]{Nome do Orientador}   % Nome do orientador aqui
%\coorientador[Prof.]{Nome do Coorientador} % Coorientador (se tiver)


%%-----------------------------------------------------------------------
% Informações sobre campus, centro, depto e curso
% \campus e \curso são obrigatórios e precisam ser preenchidos!!
% Todos os demais são opcionais (para o LaTeX), pois note que o 
% \preambulo é exigido pela Biblioteca na entrega do trabalho.
\campus{Campus I}
\centro{Centro de Ciências e Tecnologia}
\depto{Departamento de Estatística}
\curso{Curso de Bacharelado em Estatística}
\tipotrabalho{Trabalho de Conclusão de Curso}
\preambulo{
  Trabalho de Conclusão de Curso (Artigo) apresentado ao curso de Bacharelado em Estatística do Departamento de Estatística do Centro de Ciências e Tecnologia da Universidade Estadual da Paraíba, como requisito parcial à obtenção do título de bacharel em Estatística.
}


%%-----------------------------------------------------------------------
% compila o indice
\makeindex


%%-----------------------------------------------------------------------
% Ajusta numeração de capítulos, seções, etc para o formato artigo
\counterwithin{section}{chapter}


%%-----------------------------------------------------------------------
% Ajusta as margens do resumo (e abstract) pra ficar igual ao texto
\setlength\absleftindent{0cm}
\setlength\absrightindent{0cm}	


%%-----------------------------------------------------------------------
% Espaçamento simples para o formato artigo
\SingleSpacing



%%-----------------------------------------------------------------------
%%      INÍCIO DO DOCUMENTO
%%-----------------------------------------------------------------------


\begin{document}

% Retira espaço extra obsoleto entre as frases.
\frenchspacing 


%%-----------------------------------------------------------------------
% ELEMENTOS PRÉ-TEXTUAIS
\pretextual


%%-----------------------------------------------------------------------
% Capa
\imprimircapa


%%-----------------------------------------------------------------------
% Folha de rosto
\imprimirfolhaderosto


%%-----------------------------------------------------------------------
% Inserir a ficha bibliografica (depois de feita pela Biblioteca e salva 
% em PDF). Descomentar as 4 linhas abaixo.
%\begin{fichacatalografica}
%    \includepdf{ficha_catalografica.pdf}
%\end{fichacatalografica}
%\setcounter{page}{2}


%%-----------------------------------------------------------------------
% Inserir folha de aprovação
\setlength{\ABNTEXsignwidth}{8cm}    % Comp. da linha de assinaturas 
\membroum[Prof.]{Primeiro Membro (Orientador) \\ 
	Universidade Estadual da Paraíba (UEPB)}
\membrodois[Prof.]{Segundo Membro \\ 
	Universidade Estadual da Paraíba (UEPB)}
\membrotres[Profa.]{Terceiro Membro \\ 
	Universidade Estadual da Paraíba (UEPB)}
\datadefesa{DIA de MÊS de ANO}
\folhadeaprovacaoUEPB

%%-----------------------------------------------------------------------       
%%                         IMPORTANTE!!
% Após a defesa digitalize a folha de aprovação em um arquivo PDF, co-
% mente as 9 linhas acima e acrescente o PDF digitalizado descomentando
% a linha a seguir:
%\includepdf{folha_aprovacao.pdf}


%%-----------------------------------------------------------------------
%     Dedicatória (OPCIONAL)
% Comente as linhas abaixo se não quiser usar.
\newpage
\begin{dedicatoria}
   \vspace*{\fill}
	\hspace{.45\textwidth}
	\begin{minipage}{.5\textwidth}
	    Digite sua dedicatória aqui.
	\end{minipage}
\end{dedicatoria}


%%-----------------------------------------------------------------------
%      Epígrafe  (OPCIONAL)
% Comente as linhas abaixo se não quiser usar.
\newpage
\begin{epigrafe}
    \vspace*{\fill}
	\begin{flushright}
		``Digite aqui alguma frase ou algum \\
		pensamento que vc queira colocar como epígrafe.'' \\
		(Citar o Autor aqui)
	\end{flushright}
\end{epigrafe}


%%-----------------------------------------------------------------------
%        Inserir lista de ilustrações (OPCIONAL)
% Comente as linhas abaixo se não quiser usar.
\newpage
\pdfbookmark[0]{\listfigurename}{lof}
\listoffigures*
\cleardoublepage


%%-----------------------------------------------------------------------
%        Inserir lista de tabelas (OPCIONAL)
% Comente as linhas abaixo se não quiser usar.
\pdfbookmark[0]{\listtablename}{lot}
\listoftables*
\cleardoublepage


%%-----------------------------------------------------------------------
%       Para inserir lista de definições (OPCIONAL),
% Comente as linhas abaixo se não quiser usar.
\renewcommand{\listtheoremname}{LISTA DE DEFINIÇÕES}
\pdfbookmark[0]{\listtheoremname}{lod}
\begin{KeepFromToc}
	\listoftheorems[ignoreall,show=definicao]
\end{KeepFromToc}
\cleardoublepage


%%-----------------------------------------------------------------------
%       Para inserir lista de axiomas e teoremas (OPCIONAL),
% Comente as linhas abaixo se não quiser usar.
\renewcommand{\listtheoremname}{LISTA DE AXIOMAS E TEOREMAS}
\pdfbookmark[0]{\listtheoremname}{loat}
\begin{KeepFromToc}
	\listoftheorems[ignoreall,show={axioma,teorema}] %{axioma,teorema}%
\end{KeepFromToc}
\cleardoublepage


%%----------------------------------------------------------------------
%      Para inserir lista de abreviaturas e siglas (OPCIONAL)
% Comente as linhas abaixo se não quiser usar.
\begin{siglas}
  \item[ABNT] Associação Brasileira de Normas Técnicas
  \item[abnTeX] ABsurdas Normas para TeX
\end{siglas}


%%----------------------------------------------------------------------
%        Para inserir lista de símbolos (OPCIONAL)
% Comente as linhas abaixo se não quiser usar.
\begin{simbolos}
  \item[$\Gamma$] Letra grega Gama
  \item[$\Lambda$] Lambda
  \item[$\zeta$] Letra grega minúscula zeta
  \item[$ \in $] Pertence
\end{simbolos}


%%-----------------------------------------------------------------------
%         Inserir o Sumário
\pdfbookmark[0]{\contentsname}{toc}
\tableofcontents*
\cleardoublepage


%%------------------------------------------------------------------------
%        ELEMENTOS TEXTUAIS
\textual


%%-----------------------------------------------------------------------
%            Insere o Título
\begin{center}
  {\ABNTEXchapterfont\bfseries\MakeUppercase{\imprimirtitulo}} \\
  \vspace{\onelineskip}
  {\ABNTEXchapterfont\bfseries\MakeUppercase{\imprimirtituloestrangeiro}} 
\end{center}   % O título em língua estrangeira é OPCIONAL.


%%-----------------------------------------------------------------------
%             Insere os autores
\vspace{\onelineskip}
\hfill Primeiro autor\symfootnote{Aluno do curso XXX, Depto YYY, UEPB, Campina Grande, PB, email@email.com}
% Geralmente o primeiro autor é o orientando.

\hfill Segundo autor\symfootnote{Prof. XX, Depto YYYY, UEPB, Campina Grande, PB, email@email.com}
% Geralmente o segundo autor é o orientador.
%% Inserir mais autores, se necessário.


%%-----------------------------------------------------------------------
%                        RESUMOS
% O resumo em português é obrigatório, seguido da sua versão em outro 
% idioma, que pode ser inglês (mais comum), francês ou espanhol.

% resumo em português (OBRIGATÓRIO)
\setlength{\absparsep}{18pt} % ajusta espaçamento de parág. do resumo
\begin{resumo}
	Segundo a \citeonline[3.1-3.2]{NBR6028:2003}, o resumo deve ressaltar o objetivo, o método, os resultados e as conclusões do documento. A ordem e a extensão destes itens dependem do tipo de resumo (informativo ou indicativo) e do tratamento que cada item recebe no documento original. O resumo deve ser precedido da referência do documento, com exceção do resumo inserido no próprio documento. (\ldots) As palavras-chave devem figurar logo abaixo do resumo, antecedidas da expressão Palavras-chave:, separadas entre si por ponto e finalizadas também por ponto.
	
	\textbf{Palavras-chaves}: latex. abntex. editoração de texto.
	% Número máximo de 4 palavras-chave.
\end{resumo}

% resumo em inglês(OBRIGATÓRIO)
\begin{resumo}[Abstract]
	\begin{otherlanguage*}{english}
		According to \citeonline[3.1-3.2]{NBR6028:2003}, the abstract should highlight the objective, method, results and conclusions of the document. The order and extent of these items depends on the type of summary (informative or indicative) and the treatment that each item receives in the original document. The abstract must be preceded by the document reference, with the exception of the abstract inserted in the document itself. (\ldots) The keywords must appear just below the summary, preceded by the expression Keywords:, separated by point and finalized by point.
		
		\textbf{Keywords}: latex. abntex. text editoration.
	\end{otherlanguage*}
\end{resumo}
\vspace{\onelineskip}


%%------------------------------------------------------------------------
%                     Introdução
\chapter{Introdução}
\label{introducao}

Nesta parte do texto você deve digitar a introdução do seu trabalho. Neste caso, você deve incluir algumas informações importantes sobre o trabalho apresentando a importância e os objetivos que você pretende atingir.

Este documento foi escrito a partir de um ``modelo canônico'' da classe \textsf{abntex2}, bem como do pacote \textsf{abntex2cite}, para se fazer citações bibliográficas. O documento exemplifica a elaboração do trabalho de conclusão de curso e foi produzido conforme a ABNT NBR 14724:2011 \emph{Informação e documentação - Trabalhos acadêmicos - Apresentação}, conforme a documentação original do \textsf{abntex2}. O pacote \abnTeX\footnote{\url{http://abntex2.googlecode.com/}} implementa os requisitos das normas da ABNT, sendo que uma lista completa de tais normas é apresentada em \citeonline{abntex2classe}.

Dado que tal pacote não fornece nenhum modelo específico para nenhuma universidade ou instituição em particular, foi adotada aqui uma adaptação para a capa, contra-capa e folha de aprovação para o padrão adotado pela UEPB. Consulte \citeonline{abntex2-wiki-como-customizar} para mais informações sobre estas adaptações. Também existem outros manuais sobre o pacote \abnTeX\ \cite{abntex2classe,abntex2cite,abntex2cite-alf} e a classe \textsf{memoir} \cite{memoir}.

Note que todas as referências bibliográficas citadas aqui estão em um arquivo de referências do \textsf{bibtex} (com extensão \textit{.bib}); este documento pode ser usado para a construção do seu próprio arquivo de referências. Uma sugestão de programa útil para  o gerenciamento de arquivos do \textsf{bibtex} é o \textit{JabRef}\footnote{\url{http://jabref.sourceforge.net/}}.

No Anexo \ref{exemplos} são apresentados exemplos de códigos específicos para ajudar na elaboração do seu material, que foram extraídos de um modelo preparado pela equipe do \abnTeX. Note que este anexo está inserido aqui apenas a título de exemplo e deve DEVE ser removido para a elaboração do seu Trabalho de Conclusão de Curso.

\vspace{\onelineskip}


%%------------------------------------------------------------------------
%              Seção de Metodologia
\chapter{Metodologia}

Aqui deve ser digitada toda a parte metodológica do seu artigo, o que pode incluir uma breve revisão teórica (de estatística) do(s) método(s) que foram usados no trabalho. Se houve utilização de dados, é importante descrevê-los antes de iniciar a descrição teórica dos métodos adotados para análise. Note que você pode utilizar quantas seções e subseções forem necessárias para o seu trabalho.


%------------------------------------------------------------------------
\section{Base de dados}

Aqui você deve descrever o(s) banco(s) de dados utilizado(s). Se houve processo de coleta de dados, isso também deve ser descrito detalhadamente aqui.


%------------------------------------------------------------------------
\section{Métodos estatísticos}

Aqui você deve apresentar a teoria referentes ao(s) método(s) adotados.

Exemplo de definição...
\begin{definicao}
	Testando...
\end{definicao}

Exemplo de teorema...
\begin{teorema}
	Testando...
\end{teorema}

\vspace{\onelineskip}


%%------------------------------------------------------------------------
%               Seção para os resultados e discussão
\chapter{Resultados e discussão}

Neste capítulo você deve apresentar os resultados obtidos juntamente com a discussão correspondente.


%------------------------------------------------------------------------
\section{Seção 1}

Aqui é possível se usar quantas seções e subseções forem necessárias, de acordo com o trabalho.

%------------------------------------------------------------------------
\section{Seção 2}

Um exemplo de uma nova subseção para ilustrar.


%------------------------------------------------------------------------
\subsection{Subseção 1}

Agora a inclusão de uma subseção, também com o intuito de ilustrar seu no trabalho.

\vspace{\onelineskip}


%%-----------------------------------------------------------------------
%                 Conclusão
\chapter{Conclusão}

Aqui você deve apresentar as principais conclusões obtidas a partir do trabalho realizado. Você também pode (e deve) apresentar perspectivas futuras em relação ao trabalho desenvolvido.
\vspace{\onelineskip}


%------------------------------------------------------------------------
%               ELEMENTOS PÓS-TEXTUAIS
\postextual


%%-----------------------------------------------------------------------
%                Referências bibliográficas
\bibliography{modelo_Bibtex}




%%-----------------------------------------------------------------------
%%                     OBSERVAÇÃO IMPORTANTE
%
% Tudo o que vem a partir daqui é opcional e só deve ficar se for 
% usado no  trabalho.
%------------------------------------------------------------------------



%%-----------------------------------------------------------------------
%%                          Apêndices
% 
% Os Apêndices devem usados para inserir algum texto de produção SUA, 
% que não seja essencial para o trabalho principal, mas cuja inclusão 
% seja interessante.

%%------------------------------------------------------------------------
%  Inicia os apêndices
\newpage
\begin{apendicesenv}

%%-----------------------------------------------------------------------
\chapter{Primeiro apêndice}

Texto do primeiro apêndice.

\vspace{\onelineskip}

%------------------------------------------------------------------------
\chapter{Segundo apêndice}

Texto do segundo apêndice.

\end{apendicesenv}



%% ------------------------------------------------------------------------
%%                             Anexos
% 
% Os Anexos têm a mesma funcionalidade dos Apêndices, mas para a inclusão
% de algum texto que NÃO seja de SUA  autoria.

%%------------------------------------------------------------------------
%   Inicia os anexos
\begin{anexosenv}


%------------------------------------------------------------------------
\newpage
\chapter{Exemplos de códigos do \LaTeX\ e do \abnTeX}
\label{exemplos}

Este apêndice apresenta vários exemplos de códigos e comandos para se utilizar dentro do \LaTeX e do \abnTeX. Este documento foi extraído de um código de exemplo fornecido junto com os ``modelos canônicos'' do pacote \abnTeX.

%------------------------------------------------------------------------
\section{Codificação dos arquivos: UTF8}

A codificação de todos os arquivos do \abnTeX\ é \texttt{UTF8}. É necessário que você utilize a mesma codificação nos documentos que escrever, inclusive nos arquivos de base bibliográficas |.bib|.

%------------------------------------------------------------------------
\section{Citações diretas}
\label{sec-citacao}

\index{citações!diretas}Utilize o ambiente \texttt{citacao} para incluir citações diretas com mais de três linhas:
\begin{citacao}
  As citações diretas, no texto, com mais de três linhas, devem ser destacadas com recuo de 4 cm da margem esquerda, com letra menor que a do texto utilizado e sem as aspas. No caso de documentos datilografados, deve-se observar apenas o recuo \cite[5.3]{NBR10520:2002}.
\end{citacao}

Use o ambiente assim:
\begin{verbatim}
  \begin{citacao}
    As citações diretas, no texto, com mais de três linhas [...] deve-se
    observar apenas o recuo \cite[5.3]{NBR10520:2002}.
  \end{citacao}
\end{verbatim}

O ambiente \texttt{citacao} pode receber como parâmetro opcional um nome de idioma previamente carregado nas opções da classe (\autoref{sec-hifenizacao}). Nesse caso, o texto da citação é automaticamente escrito em itálico e a hifenização é ajustada para o idioma selecionado na opção do ambiente. Por exemplo:
\begin{verbatim}
  \begin{citacao}[english]
    Text in English language in italic with correct hyphenation.
  \end{citacao}
\end{verbatim}

Tem como resultado:
\begin{citacao}[english]
 Text in English language in italic with correct hyphenation.
\end{citacao}

\index{citações!simples}Citações simples, com até três linhas, devem ser incluídas com aspas. Observe que em \LaTeX as aspas iniciais são diferentes das finais: ``Amor é fogo que arde sem se ver''.

%------------------------------------------------------------------------
\section{Notas de rodapé}

As notas de rodapé são detalhadas pela NBR 14724:2011\footnote{As notas devem ser digitadas ou datilografadas dentro das margens, ficando separadas do texto por um espaço simples de entre as linhas e por filete de 5cm, a partir da margem esquerda. Devem ser alinhadas, a partir da segunda linha da mesma nota, abaixo da primeira letra da primeira palavra, de forma a destacar o expoente, sem espaço entre elas e com fonte menor \citeonline[5.2.1]{NBR14724:2011}.}\footnote{Caso uma série de notas sejam criadas sequencialmente, o \abnTeX\ instrui o \LaTeX\ para que uma vírgula seja colocada após cada número do expoente que indica a nota de rodapé no corpo do texto.}\footnote{Verifique se os números do expoente possuem uma vírgula para dividi-los no corpo do texto.}. 


%------------------------------------------------------------------------
\section{Tabelas}

\index{tabelas}A \autoref{tab-nivinv} é um exemplo de tabela construída em \LaTeX.
\begin{table}[htb]
\ABNTEXfontereduzida
\caption[Níveis de investigação]{Níveis de investigação.}
\label{tab-nivinv}
\begin{tabular}{p{2.6cm}|p{6.0cm}|p{2.25cm}|p{3.40cm}}
  %\hline
   \textbf{Nível de Investigação} & \textbf{Insumos}  & \textbf{Sistemas de Investigação}  & \textbf{Produtos}  \\
    \hline
    Meta-nível & Filosofia\index{filosofia} da Ciência  & Epistemologia &
    Paradigma  \\
    \hline
    Nível do objeto & Paradigmas do metanível e evidências do nível inferior &
    Ciência  & Teorias e modelos \\
    \hline
    Nível inferior & Modelos e métodos do nível do objeto e problemas do nível inferior & Prática & Solução de problemas  \\
   % \hline
\end{tabular}
\legend{Fonte: \citeonline{van86}}
\end{table}

Já a \autoref{tabela-ibge} apresenta uma tabela criada conforme o padrão do \citeonline{ibge1993} requerido pelas normas da ABNT para documentos técnicos e acadêmicos.
\begin{table}[htb]
\IBGEtab{%
  \caption{Um Exemplo de tabela alinhada que pode ser longa   ou curta, conforme padrão IBGE.}%
  \label{tabela-ibge}
}{%
  \begin{tabular}{ccc}
  \toprule
   Nome & Nascimento & Documento \\
  \midrule \midrule
   Maria da Silva & 11/11/1111 & 111.111.111-11 \\
  \midrule 
   João Souza & 11/11/2111 & 211.111.111-11 \\
  \midrule 
   Laura Vicuña & 05/04/1891 & 3111.111.111-11 \\
  \bottomrule
\end{tabular}%
}{%
  \fonte{Produzido pelos autores.}%
  \nota{Esta é uma nota, que diz que os dados são baseados na   regressão linear.}%
  \nota[Anotações]{Uma anotação adicional, que pode ser seguida de várias outras.}%
  }
\end{table}


%------------------------------------------------------------------------
\section{Figuras}

\index{figuras}Figuras podem ser criadas diretamente em \LaTeX, como o exemplo da \autoref{fig_circulo}.
\begin{figure}[htb]
	\caption{\label{fig_circulo}A delimitação do espaço}
	\begin{center}
	    \setlength{\unitlength}{5cm}
		\begin{picture}(1,1)
		\put(0,0){\line(0,1){1}}
		\put(0,0){\line(1,0){1}}
		\put(0,0){\line(1,1){1}}
		\put(0,0){\line(1,2){.5}}
		\put(0,0){\line(1,3){.3333}}
		\put(0,0){\line(1,4){.25}}
		\put(0,0){\line(1,5){.2}}
		\put(0,0){\line(1,6){.1667}}
		\put(0,0){\line(2,1){1}}
		\put(0,0){\line(2,3){.6667}}
		\put(0,0){\line(2,5){.4}}
		\put(0,0){\line(3,1){1}}
		\put(0,0){\line(3,2){1}}
		\put(0,0){\line(3,4){.75}}
		\put(0,0){\line(3,5){.6}}
		\put(0,0){\line(4,1){1}}
		\put(0,0){\line(4,3){1}}
		\put(0,0){\line(4,5){.8}}
		\put(0,0){\line(5,1){1}}
		\put(0,0){\line(5,2){1}}
		\put(0,0){\line(5,3){1}}
		\put(0,0){\line(5,4){1}}
		\put(0,0){\line(5,6){.8333}}
		\put(0,0){\line(6,1){1}}
		\put(0,0){\line(6,5){1}}
		\end{picture}
	\end{center}
	\legend{Fonte: os autores}
\end{figure}

Ou então figuras podem ser incorporadas de arquivos externos, como é o caso da \autoref{fig_grafico}. Se a figura que será incluída se tratar de um diagrama, um gráfico ou uma ilustração que você mesmo produza, priorize o uso de imagens vetoriais no formato PDF. Com isso, o tamanho do arquivo final do trabalho será menor, e as imagens terão uma apresentação melhor, principalmente quando impressas, uma vez que imagens vetorias são perfeitamente escaláveis para qualquer dimensão. Nesse caso, se for utilizar o Microsoft Excel para produzir gráficos, ou o Microsoft Word para produzir ilustrações, exporte-os como PDF e os incorpore ao documento conforme o exemplo abaixo. No entanto, para manter a coerência no uso de software livre (já que você está usando \LaTeX e \abnTeX), teste a ferramenta \textsf{InkScape}\index{InkScape} (\url{http://inkscape.org/}). Ela é uma excelente opção de código-livre para produzir ilustrações vetoriais, similar ao CorelDraw\index{CorelDraw} ou ao Adobe Illustrator\index{Adobe Illustrator}. De todo modo, caso não seja possível utilizar arquivos de imagens como PDF, utilize qualquer outro formato, como JPEG, GIF, BMP, etc. Nesse caso, você pode tentar aprimorar as imagens incorporadas com o software livre \textsf{Gimp}\index{Gimp} (\url{http://www.gimp.org/}). Ele é uma alternativa livre ao Adobe Photoshop\index{Adobe Photoshop}.
\begin{figure}[htb]
	\caption{\label{fig_grafico}Gráfico produzido em Excel e salvo como PDF}
	\begin{center}
	    \includegraphics[scale=0.5]{figuras/abntex2-modelo-img-grafico.pdf}
	\end{center}
	\legend{Fonte: \citeonline[p. 24]{araujo2012}}
\end{figure}

%------------------------------------------------------------------------
\subsection{Figuras em \emph{minipages}}

\emph{Minipages} são usadas para inserir textos ou outros elementos em quadros com tamanhos e posições controladas. Veja o exemplo da \autoref{fig_minipage_imagem1} e da \autoref{fig_minipage_grafico2}.
\begin{figure}[htb]
 \label{teste}
 \centering
  \begin{minipage}{0.4\textwidth}
    \centering
    \caption{Imagem 1 da minipage} \label{fig_minipage_imagem1}
    \includegraphics[scale=0.9]{figuras/abntex2-modelo-img-marca.pdf}
    \legend{Fonte: Produzido pelos autores}
  \end{minipage}
  \hfill
  \begin{minipage}{0.4\textwidth}
    \centering
    \caption{Grafico 2 da minipage} \label{fig_minipage_grafico2}
    \includegraphics[scale=0.2]{figuras/abntex2-modelo-img-grafico.pdf}
    \legend{Fonte: \citeonline[p. 24]{araujo2012}}
  \end{minipage}
\end{figure}

Observe que, segundo a \citeonline[seções 4.2.1.10 e 5.8]{NBR14724:2011}, as ilustrações devem sempre ter numeração contínua e única em todo o documento:
\begin{citacao}
  Qualquer que seja o tipo de ilustração, sua identificação aparece na parte superior, precedida da palavra designativa (desenho, esquema, fluxograma, fotografia, gráfico, mapa, organograma, planta, quadro, retrato, figura, imagem, entre outros), seguida de seu número de ordem de ocorrência no texto, em algarismos arábicos, travessão e do respectivo título. Após a ilustração, na parte inferior, indicar a fonte consultada (elemento obrigatório, mesmo que seja produção do próprio autor), legenda, notas e outras informações necessárias à sua compreensão (se houver). A ilustração deve ser citada no texto e inserida o mais próximo possível do trecho a que se refere. \cite[seções 5.8]{NBR14724:2011}
\end{citacao}

%------------------------------------------------------------------------
\section{Expressões matemáticas}

\index{expressões matemáticas}Use o ambiente \texttt{equation} para escrever expressões matemáticas numeradas:
\begin{equation}
  \forall x \in X, \quad \exists \: y \leq \epsilon.
\end{equation}

Escreva expressões matemáticas entre \$ e \$, como em $ \lim_{x \to \infty} \exp(-x) = 0 $, para que fiquem na mesma linha.

Também é possível usar cifrão duplo (\$\$) ou colchetes para indicar o início de uma expressão matemática que não é numerada.
$$
\left|\sum_{i=1}^n a_ib_i\right|
\le
\left(\sum_{i=1}^n a_i^2\right)^{1/2}
\left(\sum_{i=1}^n b_i^2\right)^{1/2}
$$

Consulte mais informações sobre expressões matemáticas em \url{https://code.google.com/p/abntex2/wiki/Referencias}.


%------------------------------------------------------------------------
\section{Enumerações: alíneas e subalíneas}

\index{alíneas}\index{subalíneas}\index{incisos}Quando for necessário enumerar os diversos assuntos de uma seção que não possua título, esta deve ser subdividida em alíneas \cite[4.2]{NBR6024:2012}:
\begin{alineas}
  \item os diversos assuntos que não possuam título próprio, dentro de uma mesma   seção, devem ser subdivididos em alíneas; 
  \item o texto que antecede as alíneas termina em dois pontos;
  \item as alíneas devem ser indicadas alfabeticamente, em letra minúscula, seguida de parêntese. Utilizam-se letras dobradas, quando esgotadas as letras do alfabeto;
  \item as letras indicativas das alíneas devem apresentar recuo em relação à   margem esquerda;
  \item o texto da alínea deve começar por letra minúscula e terminar em ponto-e-vírgula, exceto a última alínea que termina em ponto final;
  \item o texto da alínea deve terminar em dois pontos, se houver subalínea;
  \item a segunda e as seguintes linhas do texto da alínea começa sob a primeira letra do texto da própria alínea;
  \item subalíneas \cite[4.3]{NBR6024:2012} devem ser conforme as alíneas a seguir:
  \begin{alineas}
     \item as subalíneas devem começar por travessão seguido de espaço;
     \item as subalíneas devem apresentar recuo em relação à alínea;
     \item o texto da subalínea deve começar por letra minúscula e terminar em  ponto-e-vírgula. A última subalínea deve terminar em ponto final, se não houver alínea subsequente;
     \item a segunda e as seguintes linhas do texto da subalínea começam sob a primeira letra do texto da própria subalínea.
  \end{alineas}
  \item no \abnTeX\ estão disponíveis os ambientes \texttt{incisos} e \texttt{subalineas}, que em suma são o mesmo que se criar outro nível de \texttt{alineas}, como nos exemplos à seguir:
  \begin{incisos}
    \item \textit{Um novo inciso em itálico};
  \end{incisos}
  \item Alínea em \textbf{negrito}:
  \begin{subalineas}
    \item \textit{Uma subalínea em itálico};
    \item \underline{\textit{Uma subalínea em itálico e sublinhado}}; 
  \end{subalineas}
  \item Última alínea com \emph{ênfase}.
\end{alineas}


%------------------------------------------------------------------------
\section{Espaçamento entre parágrafos e linhas}

\index{espaçamento!dos parágrafos}O tamanho do parágrafo, espaço entre a margem e o início da frase do parágrafo, é definido por:
\begin{verbatim}
   \setlength{\parindent}{1.3cm}
\end{verbatim}

\index{espaçamento!do primeiro parágrafo}Por padrão, não há espaçamento no primeiro parágrafo de cada início de divisão do documento (\autoref{sec-divisoes}). Porém, você pode definir que o primeiro parágrafo também seja indentado, como é o caso deste documento. Para isso, apenas inclua o pacote \textsf{indentfirst} no preâmbulo do documento:
\begin{verbatim}
   \usepackage{indentfirst}  % Indenta o primeiro parágrafo de cada seção.
\end{verbatim}

\index{espaçamento!entre os parágrafos}O espaçamento entre um parágrafo e outro pode ser controlado por meio do comando:
\begin{verbatim}
  \setlength{\parskip}{0.2cm}  % tente também \onelineskip
\end{verbatim}

\index{espaçamento!entre as linhas}O controle do espaçamento entre linhas é definido por:
\begin{verbatim}
  \OnehalfSpacing       % espaçamento um e meio (padrão); 
  \DoubleSpacing        % espaçamento duplo
  \SingleSpacing        % espaçamento simples	
\end{verbatim}

Para isso, também estão disponíveis os ambientes:
\begin{verbatim}
  \begin{SingleSpace} ...\end{SingleSpace}
  \begin{Spacing}{hfactori} ... \end{Spacing}
  \begin{OnehalfSpace} ... \end{OnehalfSpace}
  \begin{OnehalfSpace*} ... \end{OnehalfSpace*}
  \begin{DoubleSpace} ... \end{DoubleSpace}
  \begin{DoubleSpace*} ... \end{DoubleSpace*} 
\end{verbatim}

Para mais informações, consulte \citeonline[p. 47-52 e 135]{memoir}.


% ------------------------------------------------------------------------
\section{Inclusão de outros arquivos}\label{sec-include}

É uma boa prática dividir o seu documento em diversos arquivos, e não apenas escrever tudo em um único. Para incluir diferentes arquivos em um arquivo principal, de modo que cada arquivo incluído fique em uma página diferente, utilize o comando:
\begin{verbatim}
   \include{documento-a-ser-incluido}      % sem a extensão .tex
\end{verbatim}

Para incluir documentos sem quebra de páginas, utilize:
\begin{verbatim}
   \input{documento-a-ser-incluido}      % sem a extensão .tex
\end{verbatim}


%------------------------------------------------------------------------
\section{Compilar o documento \LaTeX}

Geralmente os editores \LaTeX, como o TeXstudio\footnote{\url{http://texstudio.sourceforge.net}} e o Texmaker\footnote{\url{http://www.xm1math.net/texmaker/}}, entre outros, compilam os documentos automaticamente, de modo que você não precisa se preocupar com isso.

No entanto, você pode compilar os documentos \LaTeX usando os seguintes comandos, que devem ser digitados no \emph{Prompt de Comandos} do Windows ou no \emph{Terminal} do Mac ou do Linux:
\begin{verbatim}
 pdflatex ARQUIVO_PRINCIPAL.tex
 bibtex ARQUIVO_PRINCIPAL.aux
 makeindex ARQUIVO_PRINCIPAL.idx 
 makeindex ARQUIVO_PRINCIPAL.nlo -s nomencl.ist -o ARQUIVO_PRINCIPAL.nls
 pdflatex ARQUIVO_PRINCIPAL.tex
 pdflatex ARQUIVO_PRINCIPAL.tex
\end{verbatim}


%------------------------------------------------------------------------
\section{Remissões internas}

Ao nomear a \autoref{tab-nivinv} e a \autoref{fig_circulo}, apresentamos um exemplo de remissão interna, que também pode ser feita quando indicamos a \autoref{introducao}, que tem o nome \emph{\nameref{introducao}}. O número da seção indicada é \ref{introducao}, que se inicia à \autopageref{introducao}\footnote{O número da página de uma remissão pode ser obtida também assim: \pageref{introducao}.}. Veja a \autoref{sec-divisoes} para outros exemplos de remissões internas entre seções, subseções e subsubseções.

O código usado para produzir o texto desta seção é:
\begin{verbatim}
 Ao nomear a \autoref{tab-nivinv} e a \autoref{fig_circulo}, apresentamos
 um exemplo de remissão interna, que também pode ser feita quando 
 indicamos o \autoref{introducao}, que tem o nome 
 \emph{\nameref{introducao}}. O número do capítulo indicado é 
 \ref{introducao}, que se inicia à \autopageref{introducao}\footnote{O 
 número da página de uma remissão pode ser obtida também assim:
 \pageref{introducao}.}. Veja a \autoref{sec-divisoes} para outros 
 exemplos de remissões internas entre seções, subseções e subsubseções.
\end{verbatim}


%------------------------------------------------------------------------
\section{Divisões do documento: seção}\label{sec-divisoes}

Esta seção testa o uso de divisões de documentos. Esta é a \autoref{sec-divisoes}. Veja a \autoref{sec-divisoes-subsection}.

\subsection{Divisões do documento: subseção}\label{sec-divisoes-subsection}

Isto é uma subseção. Veja a \autoref{sec-divisoes-subsubsection}, que é uma \texttt{subsubsection} do \LaTeX, mas é impressa chamada de ``subseção'' porque no Português não temos a palavra ``subsubseção''.

\subsubsection{Divisões do documento: subsubseção}
\label{sec-divisoes-subsubsection}

Isto é uma subsubseção.

\subsubsection{Divisões do documento: subsubseção}

Isto é outra subsubseção.

\subsection{Divisões do documento: subseção}\label{sec-exemplo-subsec}

Isto é uma subseção.

\subsubsection{Divisões do documento: subsubseção}

Isto é mais uma subsubseção da \autoref{sec-exemplo-subsec}.


\subsubsubsection{Esta é uma subseção de quinto nível}\label{sec-exemplo-subsubsubsection}

Esta é uma seção de quinto nível. Ela é produzida com o seguinte comando:
\begin{verbatim}
  \subsubsubsection{Esta é uma subseção de quinto nível}
  \label{sec-exemplo-subsubsubsection}
\end{verbatim}

\subsubsubsection{Esta é outra subseção de quinto nível}\label{sec-exemplo-subsubsubsection-outro}

Esta é outra seção de quinto nível.

\paragraph{Este é um parágrafo numerado}\label{sec-exemplo-paragrafo}

Este é um exemplo de parágrafo numerado. Ele é produzido com o comando de parágrafo:
\begin{verbatim}
  \paragraph{Este é um parágrafo nomeado}
  \label{sec-exemplo-paragrafo}
\end{verbatim}

A numeração entre parágrafos numeradaos e subsubsubseções são contínuas.

\paragraph{Este é outro parágrafo numerado}\label{sec-exemplo-paragrafo-outro}

Este é outro parágrafo numerado.

%------------------------------------------------------------------------
\section{Este é um exemplo de nome de seção longo. ele deve estar alinhado à esquerda e a segunda e demais linhas devem iniciar logo abaixo da primeira palavra da primeira linha}

Isso atende à norma \citeonline[seções de 5.2.2 a 5.2.4]{NBR14724:2011} e \citeonline[seções de 3.1 a 3.8]{NBR6024:2012}.

%------------------------------------------------------------------------
\section{Diferentes idiomas e hifenizações}
\label{sec-hifenizacao}

Para usar hifenizações de diferentes idiomas, inclua nas opções do documento o nome dos idiomas que o seu texto contém. Por exemplo (para melhor visualização, as opções foram quebras em diferentes linhas):
\begin{verbatim}
  \documentclass[
     12pt,
     openright,
     twoside,
     a4paper,
     english,
     french,
     spanish,
     brazil
  ]{abntex2}
\end{verbatim}

O idioma português-brasileiro (\texttt{brazil}) é incluído automaticamente pela classe \textsf{abntex2}. Porém, mesmo assim a opção \texttt{brazil} deve ser informada como a última opção da classe para que todos os pacotes reconheçam o idioma. Vale ressaltar que a última opção de idioma é a utilizada por padrão no documento. Desse modo, caso deseje escrever um texto em inglês que tenha citações em português e em francês, você deveria usar o preâmbulo como abaixo:
\begin{verbatim}
  \documentclass[
      12pt,
      openright,
      twoside,
      a4paper,
      french,
      brazil,
      english
  ]{abntex2}
\end{verbatim}

A lista completa de idiomas suportados, bem como outras opções de hifenização, estão disponíveis em \citeonline[p.~5-6]{babel}.

Exemplo de hifenização em inglês\footnote{Extraído de: \url{http://en.wikibooks.org/wiki/LaTeX/Internationalization}}:

\begin{otherlanguage*}{english}
\textit{Text in English language. This environment switches all language-related definitions, like the language specific names for figures, tables etc. to the other language. The starred version of this environment typesets the main text according to the rules of the other language, but keeps the language specific string for ancillary things like figures, in the main language of the document. The environment hyphenrules switches only the hyphenation patterns used; it can also be used to disallow hyphenation by using the language name `nohyphenation'.}
\end{otherlanguage*}

Exemplo de hifenização em francês\footnote{Extraído de: \url{http://bigbrowser.blog.lemonde.fr/2013/02/17/tu-ne-tweeteras-point-le-vatican-interdit-aux-cardinaux-de-tweeter-pendant-le-conclave/}}:

\begin{otherlanguage*}{french}
\textit{Texte en français. Pas question que Twitter ne vienne faire une concurrence déloyale à la traditionnelle fumée blanche qui marque l'élection d'un nouveau pape. Pour éviter toute fuite précoce, le Vatican a donc pris un peu d'avance, et a déjà interdit aux cardinaux qui prendront part au vote d'utiliser le réseau social, selon Catholic News Service. Une mesure valable surtout pour les neuf cardinaux – sur les 117 du conclave – pratiquants très actifs de Twitter, qui auront interdiction pendant toute la période de se connecter à leur compte.}
\end{otherlanguage*}

Pequeno texto em espanhol\footnote{Extraído de: \url{http://internacional.elpais.com/internacional/2013/02/17/actualidad/1361102009_913423.html}}:

\foreignlanguage{spanish}{\textit{Decenas de miles de personas ovacionan al pontífice en su penúltimo ángelus dominical, el primero desde que anunciase su renuncia. El Papa se centra en la crítica al materialismo}}.

O idioma geral do texto por ser alterado como no exemplo seguinte:
\begin{verbatim}
  \selectlanguage{english}
\end{verbatim}

Isso altera automaticamente a hifenização e todos os nomes constantes de referências do documento para o idioma inglês. Consulte o manual da classe \cite{abntex2classe} para obter orientações adicionais sobre internacionalização de documentos produzidos com \abnTeX.

A \autoref{sec-citacao} descreve o ambiente \texttt{citacao} que pode receber como parâmetro um idioma a ser usado na citação.


%------------------------------------------------------------------------
\section{Consulte o manual da classe \abnTeX}

Consulte o manual da classe \textsf{abntex2} \cite{abntex2classe} para uma referência completa das macros e ambientes disponíveis. 

Além disso, o manual possui informações adicionais sobre as normas ABNT observadas pelo \abnTeX\ e considerações sobre eventuais requisitos específicos não atendidos, como o caso da \citeonline[seção 5.2.2]{NBR14724:2011}, que especifica o espaçamento entre os capítulos e o início do texto, regra propositalmente não atendida pelo presente modelo.


%------------------------------------------------------------------------
\section{Referências bibliográficas}

A formatação das referências bibliográficas conforme as regras da ABNT são um dos principais objetivos do \abnTeX. Consulte os manuais \citeonline{abntex2cite} e \citeonline{abntex2cite-alf} para obter informações sobre como utilizar as referências bibliográficas.

%------------------------------------------------------------------------
\subsection{Acentuação de referências bibliográficas}

Normalmente não há problemas em usar caracteres acentuados em arquivos bibliográficos (\texttt{*.bib}). Porém, como as regras da ABNT fazem uso quase abusivo da conversão para letras maiúsculas, é preciso observar o modo como se escreve os nomes dos autores. Na ~\autoref{tabela-acentos} você encontra alguns exemplos das conversões mais importantes. Preste atenção especial para `ç' e `í' que devem estar envoltos em chaves. A regra geral é sempre usar a acentuação neste modo quando houver conversão para letras maiúsculas.
\begin{table}[htbp]
\caption{Tabela de conversão de acentuação.}
\label{tabela-acentos}
\begin{center}
\begin{tabular}{ll}\hline\hline
acento & \textsf{bibtex}\\
à á ã & \verb+\`a+ \verb+\'a+ \verb+\~a+\\
í & \verb+{\'\i}+\\
ç & \verb+{\c c}+\\
\hline\hline
\end{tabular}
\end{center}
\end{table}


%------------------------------------------------------------------------
\section{Precisa de ajuda?}

Consulte a FAQ com perguntas frequentes e comuns no portal do \abnTeX: \url{https://code.google.com/p/abntex2/wiki/FAQ}.

Inscreva-se no grupo de usuários \LaTeX: \url{http://groups.google.com/group/latex-br}, tire suas dúvidas e ajude outros usuários.

Participe também do grupo de desenvolvedores do \abnTeX: \url{http://groups.google.com/group/abntex2} e faça sua contribuição à ferramenta.
\vspace{\onelineskip}


%------------------------------------------------------------------------
\chapter{Segundo anexo}

Texto do segundo anexo.

\end{anexosenv}


%%-----------------------------------------------------------------------
%                          Agradecimentos (OPCIONAL)
% Comente as linhas abaixo se não quiser usar.
\newpage
\begin{agradecimentos}
	Digite aqui os seus agradecimentos pessoais. Pode ser para pessoas,
	instituições, etc.
\end{agradecimentos}


%%-----------------------------------------------------------------------
%%                      INDICE REMISSIVO
%%-----------------------------------------------------------------------
\phantompart
\printindex


\end{document}
